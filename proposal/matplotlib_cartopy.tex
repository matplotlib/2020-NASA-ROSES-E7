%!TEX TS-program = XeLaTeX
\documentclass[12pt]{article}
\usepackage[top=1in, bottom=1in, left=1in, right=1in]{geometry}

%%%
%% Needed for fonts in xelatex to work
%%%
% NOTE: I actually use XeLaTeX, which allows me to get the fonts
% exactly the way that I want them. For proposals, this means
% I can use Times New Roman instead of the default Computer
% Modern. I actually like Computer Modern, but since Arial
% is what the solicitation suggests there's no point in throwing off a reviewer
% with an unexpected font, particularly one with such a
% polarizing reaction in readers. Never upset the reviewrers, I
% always say.
%
% What's the point of this bit of rambling? If you do not want to use
% XeLaTeX and would rather stick to good old LaTeX, then you
% need to comment out the next few lines of font packages and
% font commands.
%
% If you want to use XeLaTeX but want different fonts, then you
% just need to change the name in the argument for \setmainfont.
% Make sure that the font you use is loaded on
% your machine and your TeX distribution knows how to find it.
% See Google if you need to learn more about this.
%
\usepackage{fontspec}
\setmainfont{Arial}

%%%
%% Packages that I use on a regular basis.
%%%
% Of course, you are likely to need some math typesetting so these
% three packages have you covered.
\usepackage{amssymb}
\usepackage{amsmath}
\usepackage{latexsym}
% I use color, graphicx, and epstopdf to read in PDFs for my figures.
\usepackage{color}
\usepackage{graphicx}
% \usepackage{epstopdf}
% I don't remember why threeparttable and setspace is here. Inertia.
\usepackage{threeparttable}
\usepackage{setspace}
%%%
%% Some packages to handle the figures and captions
%%%
\usepackage[labelfont=bf]{caption}
\usepackage{subcaption}
\usepackage{wrapfig}

%%%
%% Packages and settings for my bibliography.
%%%
% apa_with_doi is a style I created to keep DOI in the bibliography
% but strip out URLs. There are a lot of other styles you can
% find for natbib. Again, Google is your friend.
% Author name and year references, i.e., Author (year):
%\usepackage{natbib}
%\bibliographystyle{apa_with_doi}
% Numbered references:
\usepackage[numbers,super]{natbib}
\bibliographystyle{unsrtnat}


%%%
%% Packages and commands to build my table of contents (TOC).
%%%
%% The trick was getting the References included properly.
%% Also, some of my table of contents entry have no page number
%% because those pages are generated separately by my institute.
%% Nothing to be done about that. You may or may not have the
%% same problem, so you may or may not have to tweak this.
\usepackage[nottoc,numbib]{tocbibind}
\renewcommand{\tocbibname}{References}
\usepackage{tocloft}
\renewcommand{\cftsecleader}{\cftdotfill{\cftdotsep}}

%%%
%% These commands get the spacing around the title and section titles right.
%%%
% I tightened up the spacing. The LaTeX default is just too roomy.
% This spacing is still clean and legible, just not so free with the
% whitespace between sections.
%
% First the title.
\usepackage{titling}
\setlength{\droptitle}{-50pt}
\pretitle{\begin{center}\Large\bfseries\vspace{0ex}}%
\posttitle{\end{center}\Large\vspace{-2ex}}%
\preauthor{\begin{center}\large}%
\postauthor{\end{center}\large\vspace{-3ex}}%
\predate{\begin{center}\large}%q
\postdate{\end{center}\large\vspace{-6ex}}%
% Now the section headings.
\usepackage[noindentafter]{titlesec}
\titleformat{\section}{\large\bfseries}{\thesection}{1em}{}
\titlespacing{\section}{0pt}{18pt plus 2pt minus 2pt}{4pt plus 2pt minus 2pt}[0pt]
\titlespacing{\subsection}{0pt}{16pt plus 2pt minus 2pt}{4pt plus 2pt minus 2pt}[0pt]
\titlespacing{\subsubsection}{0pt}{14pt plus 2pt minus 2pt}{4pt plus 2pt minus 2pt}[0pt]

%%%
%% These commands get the lists to work the way that I want them to.
%%%
% i.e. I want less space wrapping around the list.
\usepackage{enumitem}
\setlist{nolistsep}
\setlist[2]{noitemsep}
\setlist[1]{noitemsep}

%%%
%% Commands for making the tables.
%%%
\usepackage{booktabs}
\usepackage{multirow}
\usepackage{array}


%%%
%%% Formatting urls
%%%
\usepackage{url}
\urlstyle{rm}

%% The lineno packages adds line numbers. Start line numbering with
%% \begin{linenumbers}, end it with \end{linenumbers}. Or switch it on
%% for the whole article with \linenumbers after \end{frontmatter}.
\usepackage{lineno}

%% In order to have a caption to the side of a figure or table, use the
%% 'sidecap' package.
\usepackage[rightcaption]{sidecap}
\sidecaptionvpos{figure}{t}

\usepackage{wrapfig}


%% For more control of the enumeration environment (lists with numbers)
%% use the enumitem package.
%\usepackage{enumitem}

%% Also, to reset the numbering of enumerate, use the following:
%\setenumerate[0]{label=\alph*.}

% To deal with figures all alone on a page.
\renewcommand{\floatpagefraction}{.8}%

% To use symbols for the footnotes:
\renewcommand{\thefootnote}{\fnsymbol{footnote}}

% set up the page numbers as 1-N, 2-N, ...
\numberwithin{page}{section}
\renewcommand{\thepage}{\thesection-\arabic{page}}

% https://tex.stackexchange.com/questions/210871/latex-page-numbering-by-section
%this does not seem to work, just hard code it :(
% not sure if there is something else in this template that is breaking it
% or things have changed in the last 6 years?
%\usepackage{etoolbox}
%\makeatletter
%% Make sure that page starts from 1 with every \section
%\patchcmd{\@sect}% <cmd>
%  {\protected@edef}% <search>
%  {\def\arg{#1}\def\arg@{section}%
%   \ifx\arg\arg@\stepcounter{page}\fi%
%   \protected@edef}% <replace>
%  {}{}% <success><failure>
%\makeatother

%% Finally, we get to the document.
\begin{document}
\title{Improving the Foundations and Maintenance of Matplotlib and CartoPy}
\author{Dr. Thomas A Caswell\\Dr. Ryan May}
\date{}
\maketitle

% First, let's get that TOC in there. NASA likes it.
\setcounter{tocdepth}{2}
\tableofcontents
\thispagestyle{empty}
% Let's leave this TOC alone on this page and start a new one for
% proposal body.
\newpage

\section{Scientific/Technical/Management (S/T/M)}
% Let's reset the page counter.
\setcounter{page}{1}
% the subsection are a combination of the lines labeled "Content" in
% Table 1 (on ROSES-20 SoS-51) and the text in E.7.3 (on E.7-2 -
% E.7-3) describing what needs to be in the proposal.

% not sure that this order is right, but I think we need to hit all of
% these points.  Maybe want to rename the section headings or merge
% some of them?
\subsection{description of software and relevance to SMD}

Billions of dollars of SMD funding, across the 4 divisions, critically
relies on the scientific Python ecosystem (SPE) for data analysis,
scientific computation and visualization, including flagship missions
like Hubble and JWST.  The Scientific Python Ecosystem is a loosely
defined community of projects and programmers with the common goal of
advancing the use of the Python programming language in Science.  This
development is largely volunteer work, or work that is sponsored
implicitly by specific science projects.  SPE, shown with a rough
schematic in Figure \ref{fig:ecosystem}, has a core of general purpose
domain-agnostic tools, like NumPy\cite{Harris2020} and
SciPy\cite{Virtanen2020}, with concentric rings of increasingly domain
specific tools, like AstroPy\cite{robitaille2013astropy} and
SunPy\cite{sunpy_community2020}, extending the core.  This layered
approach gives scientists both convenient high level tools and direct
access to the underlying libraries when needed.
Matplotlib\cite{Hunter:2007} is the fundamental data visualization
library for the SPE.


CartoPy is a Matplotlib extension library that brings support for
mapping applications to Matplotlib, providing support for plotting
using a variety of map projections (both terrestrial and non) as well
as providing out-of-the-box support for many terrestrial map
features. CartoPy is currently the only Matplotlib-based plotting
library for mapping applications being actively developed in the
Python ecosystem. CartoPy is frequently used for earth science
applications, including many uses of NASA Earth science datasets.

\begin{wrapfigure}{r}{0.5\textwidth}
  \includegraphics[width=0.45\textwidth]{scipy-ecosystem}
  \caption{A schematic of the Scientific Python ecossytem.  At the
    center we have the Python language itself with concentric rings of
    domain agnostic to domain specific libraries.  Both AstroPy (top
    left) and SunPy (top right), used in the astrophysics and
    heliophysics divisions respectively, rely on Matplotlib. Cartopy,
    which is not shown, would be placed in the outer [CHECK] ring.
    Credit: Jake van der Plas, "The Unexpected Effectiveness of Python
    in Science", PyCon 2017}
  \label{fig:ecosystem}
\end{wrapfigure}


The initial commits in the Matplotlib history date to early 2003 and
the initial work was done in 2001-2002. Matplotlib has been actively
developed and maintained by a vibrant, primarily volunteer, community
over the last 17 years.  Matplotlib has over 1,300 individual
contributors to the code base, estimated to have over a million users,
is in top 100 most downloaded Python packages from pypi, and is packaged
by every major linux distribution.

The most common visualizations in a domain need to be fluid for the
end-practitioners, with the ``obvious'' customization options
exposed. Much of the domain-specific specialization is carried in the
structure, semantics and assumptions of the data, and in the standard
visualizations of the domain. These specializations can vary widely,
in contradictory ways, between domains. Because no high-level API can
simultaneously satisfy all of the visualization needs, there has
developed a rich ecosystem of domain specific plotting tools including
including yt, astropy, ArviZ, xarray, cartopy, astropy,
fast\_histogram, and metpy [tune to be more NASA specific?].

Matplotlib \cite{Hunter:2007} is an established open source plotting
library with a BSD-derived license that is currently used through out
both the SMD science community and the wider scientific community for
both publication quality figures and for

Matplotlib has an API for both quick plotting, as required from
exploratory data analysis, and an API that gives full control the
visualization for fine-tuning plots for publication.  Matplotlib also
provides a framework to develop GUI independent interactive data
exploration tools and animation.

While large parts of the scientific computing community rely on the
SPE, the open nature of these infrastructure tools makes measuring
their influence hard.  We expect that a large fraction of NASA
SMD-sponsored projects rely on this shared infrastructure, but this is
hard to prove in practice.  In particular, scientific work does not
usually cite the software that was used for computation, and usage and
download statistics are hard to gather for widely used open source
packages.  Even though citation counts are likely to vastly
under-represent the scientific use of these packages,
Matplotlib~\cite{Hunter:2007} over 6000 citations while a
common reference of NumPy\cite{walt2011numpy} has over 3200 citations
[TODO update the citation counts]. These counts already illustrate a
problem in measuring usage using citations, as every user of
Matplotlib also uses NumPy, but Matplotlib has twice as
many citations.   AstroPy~\cite{robitaille2013astropy} for
astronomy (about 2000 citations) often have high citation counts
compared to the core packages like NumPy, SciPy and Matplotlib that
they are built upon.


\subsection{Objectives and Significance}
We propose a four prongs of work:

\begin{itemize}
\item Overhaul how cartopy binds to PROJ
\item Fully Document and test existing unit handling code in Matplotlib.
\item Improve the user experience of working with unit-aware data.
\item General maintenance of Matplotlib and cartopy
\end{itemize}

These tasks are the sort of work that is least suited to volunteer
work: they either require significant dedicated effort or do not
attract volunteer effort.

Overhauling the cartopy binding to PROJ and improving the user
experience of unit-aware plotting are project that are too big to
realistically carried about by a volunteer.  They will require a
significant fraction of an FTE of work and dedicated blocks of time
due to the scope of the work.  Documenting and testing the current
functionality is important work that we have not been able to do with
volunteer effort because it is tedious thankless work that does not
produce the same dopamine hit that fixing a bug or implementing a new
feature does.

Maintenance is inherently reactive, we do not know about new bugs,
user issues, or breaking changes in upstream packages until they are
reported to us.  While this critical work can be done by volunteers,
if we want to commit to a service level we need paid developer time to
commit to that.


\subsubsection{Cartopy guts}
CartoPy's mapping capabilities leverages the widely-used PROJ library
for mapping projections and the GEOS library for geometry
applications. Currently, CartoPy maintains custom code to interface
Python with these C-based libraries. As part of this work, we propose
porting Cartopy to instead use existing libraries that provide Python
bindings for these libraries, PyPROJ and PyGEOS. Not only will this
reduce CartoPy's overall amount of code and enhance maintainability,
but it will also address two of the biggest challenges that occur in
packaging CartoPy for its user-base.


\subsubsection{Unit work}

Units are fundamental to science, however most numerical software is
unit-naive.  It the user's responsibility to keep track of and
correctly convert data to commensurate units prior to any computation.
Failing to do so leads to the most insidious of bugs: everything
``works'' but silently gives the wrong answer.  There are several
high-level libraries for unit-aware data structures in Python and
Matplotlib supports their use for unit-aware plotting.  The initial
development of this capability was supported by NASA and it is
currently used to support spaceflight operations by Monte, JPL's
mission design and navigation software system.

However, this support is under documented making it difficult for
users to understand how to fully make use of the capability and for
developers to extend it.  Further, because we do not have full test
coverage of all plotting functions to verify that they correctly
handle units we have had a number of regressions in the functionality.
We will propose to write thorough user and technical guides to working
with unit-aware data in Matplotlib, add 100\% test coverage of unit
support in our plotting routines, and fix any issues and
inconsistencies discovered in this process.  Once we have hardened the
existing functionality to make working with unit-full data easier.

Almost of all of the plotting functions in Matplotlib take in
container objects (e.g. lists, arrays) of a scalar values
(e.g. floats, ints).  The first step of the plotting routines is to
inspect the input and determine if either the container or the scalar
types are ones that we know to be unit-full.  If we detect a data
structure with units attached, we then look up the correct conversion
function to map to unit-naive values and the configure the Axis to
place ticks and format the tick labels correctly.  In addition to the
types that Matplotlib registers by default, this is the mechanism by
which we support plotting date-time and string-categorical data, there
is an API for users and third-party to register their data structures.

The first task is to document the full details, both at the API and
conceptual level, how the dispatch system works and where the
resulting state is stored internally.  This documentation will be
aimed at other Matplotlib developers and developers of third-party
libraries that want to make use of the unit system.  Understanding the
current state of the library and laying out how it should work will be
the foundation of the rest of the unit work

The second task is to write several user-facing tutorials making use
of the unit-full types we support by default and well-known
down-stream unit libraries such as
\texttt{unyt}\footnote{\url{https://unyt.readthedocs.io/en/stable/}}
and
\texttt{pint}\footnote{\url{https://pint.readthedocs.io/en/stable/}}.
In parallel we will reach out to the community to find any additional
users, such as the Monte developers, who depend on the unit machinery
to make sure that the examples are representative of how they actually
use the functionality.  As with the developer documentation this will
serve as foundation for how it is expected to work.

The third task is to ensure that all of the plotting methods support
unit-full data in a uniform way, with both the initially passed and
updated data.  During this process we expect to find both bugs and
inconsistencies in how the unit-full data is handled across the
library.  We will make use of the documentation developed as part of
the first two tasks to guide how we resolve any inconsistencies in the
behavior.

The forth task is to extend the unit machinery to extend the
functionality and improve the user experience.  There are a number of
long standing requests, such as the ability to uniformly change the
displayed unit on an axis without resetting the data (to the degree
supported by the underlying unit library).  Implementing these
features may require significant refactoring of the internal
representation of the user supplied and converted data.  To be
confident that we are doing is this a way that does not introduce more
regressions we will rely on the documentation and testing work.  This
work will likely compliment and build on the ongoing work to redesign
the Artist layer.


\subsubsection{General maintenance}

Matplotlib and Cartopy are community-driven projects, but we have
grown to the point where we need developers with the time to organize,
plan, and make decisions.

Pull Requests (PRs) and Issues are submitted faster than they can be
reviewed; Matplotlib has accumulated about 300 open PRs 1300 open
Issues due to this imbalance, while CartoPy currently stands at 40
open PRs and 229 open Issues.  This imbalance is not due lack of
activity, in 2020 Matplotlib resolved 125-200 PRs and 75-125 Issues a
month but only reduced the PR backlog by 70 open PRs and making no progress
on reducing the Issue backlog.

There may be critical bug reports or insightful feature requests among
the former, while among the latter are useful contributions or bug
fixes that would improve the libraries for direct users and downstream
packages.  The backlog is discouraging for new and occasional
contributors and distracting for core developers. The resources we
will request will help to significantly reduce, but not eliminate,
this backlog.

To maintain Matplotlib's and Cartopy's health we need to:
\begin{itemize}[noitemsep]
\item fix critical bugs and regressions;
\item triage the backlog of Issues and PRs in terms of topic, difficulty, and urgency and promptly triage newly opened Issues and PRs;
\item maintain backward compatibility and extensively document intentional changes;
\item on-board new contributors to sustain and diversify developer team;
\item and manage discussions about proposed enhancements, features, and API changes.
\end{itemize}

The requested support for developers is intended to complement and
facilitate, not replace, crucial volunteer work.  We aim to better
co-ordinate and nurture their efforts, with the goal of growing and
sustaining a diverse community of volunteer and paid expert
contributors.

For 10 months in 2020 we had a full time RSE working on Matplotlib who
had a measurable impact on both the rate of PR review, the rate of
issue resolution, and fixed a number of long standing bugs.  He has
had the time and bandwidth to address a number of long-standing issues
such as [TODO highlights from Elliott].  While Elliott has made sure
we made progress on the PR backlog and held steady on the issue
backlog.

There are a number of unglamorous tasks that are required to ``keep
the lights on'' on a Python package include things like maintaining
the CI, release management, and tracking down platform specific bugs.

Given the backlog of Issues and Pull Requests, even if supported fully
and assuming that our current level of funding in maintained, the do
not expect to eliminate the backlog in the funding period.

\subsection{Perceived impact of work}
\begin{enumerate}
\item increased reliability of unit-aware plotting
\item improvement in general reliability of software
\item improved responsively to issues
\item easier installation and deployment
\end{enumerate}
\subsection{Relevance to program element}

TODO make this flow...at all...

Matplotlib is used through out all of the program elements of the SMD.
In Earth observation Matplotlib has been used to study thunderstorms
\cite{https://doi.org/10.1002/2016JD025299,https://doi.org/10.1029/2019JD030874},
seasonal ocean winds \cite{https://doi.org/10.1002/2017JD027516} and
tropical storms \cite{Lang_2020}.  Matplotlib was used in the first
science paper from the Parker Solar probe\cite{Bale2019}.  Matplotlib
has been used as part of the Martian science program in both an
orbiter \cite{https://doi.org/10.1029/2019JE006188} and
rover\cite{https://doi.org/10.1002/2016EA000219} contexts.
Historically, Matplotlib was used as part of ground operations from
the Phoenix Lander.  Matplotlib was used on data from Kepler and K2
missions to study Trojan asteroids\cite{Nixon_2019} and Titan
\cite{Ryan_2017,2019PASP..131h4505P}.
Matplotlib is used for visualization of scheduling, safety+constraint checks, and telemetry by the Swift science operations team \cite{swift_ops,2020ApJ...900...35T}.
Both the Hubble and James Web Space Telescopes data processing
developed by Space Telescope Science Institute rely on Matplotlib [CITE NEEDED].
Matplotlib is used to support spaceflight operations by Monte, JPL's
mission design and navigation software system [CITE NEEDED].
Matplotlib is being used in the New Horizons Kuiper belt extended mission \cite{Porter_2018}.  Matplotlib
has been used for fundamental research on graphene \cite{PhysRevLett.120.236802} and
work on nuclear rockets \cite{leu_cerment}.

Need more examples of SMD using Cartopy!

Matplotlib supports plotting unit-aware data structures which have
been used to support spaceflight operations by Monte, JPL's mission
design and navigation software system. (TODO expand this!)

\subsection{Technical approach and methodology}

our standard practice!

\subsection{sources of uncertainty}

Units and cartopy overhaul may be more work than expected when we
start digging in.

Maintenance is low risk.  We know there is largo volume of
individually small tasks to be done and have demonstrated that paid
developers increases the rate at which the issues / PRs are resolved.

\subsection{mitigation to risk}

Will not regress from functional status quo, all work will be an
improvement but maybe not as much as we plan for.

\subsection{roles of team members}
\begin{enumerate}
\item Caswell: PI + maintenance + community/governance work
\item May: Cartopy refactor + unit expertise + maintenance
\item RSE: maintenance + lead unit unit testing
\end{enumerate}

\subsection{workplan with milestones}

\subsection{Project Management}
\subsubsection{Governance}
Matplotlib is a NumFOCUS Fiscally Sponsored Project.  The governance
is specified by
\url{https://github.com/matplotlib/governance/blob/master/governance.MD}.
The project has a Project Lead (Caswell) who is the final authority in
all decisions, however when possible all decisions are made by
community consensus.  In addition to the Project Lead, there is a
formalized Steering Council which is responsible for the overall
direction of the project, and several Deputy Project Leads who have
day-to-day technical responsibilities.

TODO need to cover history of project lead changing



\subsubsection{License}

BSD-derived and LGPL

\subsubsection{sustainability metrics}
\begin{enumerate}
\item Continue Regular releases (mpl minor every 6mo, patch every 2mo
  or as needed).
\item Increase number of new regular contributors
\item Total number of outstanding open PRs / Issues reduced
  \begin{enumerate}
  \item caveats on how issues/PRs are very varied in time to resolve
  \end{enumerate}
\end{enumerate}

\subsubsection{collaboration with related projects}

Matplotlib exists with in the broader SPE ecosystem and has well
established relationships with the other projects in the ecosystem.
None of the projects in the Scientific Python ecosystem existing in a
vacuum; most users use more than one of the projects.  While there are
developer communities around each of the projects, there is also a
broader community across the projects.

Communication between the projects can be online, such as reporting
issues to each other's issue trackers, asking questions on mailing
lists or discussions forums, and submitting patches or in-person at
meetings and conferences.

Almost all of the Matplotlib contributors, are domain experts in their
primary domain and use Matplotlib for their research.

The strongest relationships, both historical and current, are with
projects where we have shared developers.  For example Ryan May and
Elliot sales de Andre are both core contributors to Matplotlib and
Cartopy.  Micheal Droettboom, the previous Matplotlib Project lead,
was also a core developer on Astropy.



\subsubsection{inclusive practices}

Matplotlib strives to be an inclusive and open project and have
adopted a Code of Conduct
\url{https://github.com/matplotlib/matplotlib/blob/master/CODE_OF_CONDUCT.md}. Anyone
who is willing to contribute to the project should be able to do.  It
is important for everyone working on the project to feel safe to make
mistakes.

One challenge of being an open community developed project is that we do not
have reliable demographics on a vast majority of our contributors.

We have recently started two efforts to improve the development and
retention of new contributors: an ``incubator'' channel on gitter and
a Triage Team.

The hardest part of getting started to contributing to open source
projects is can be simply getting started.  The incubator is a
semi-closed chat room where new contributors can get support on any
aspect of contributing to Matplotlib.  This include the technical
aspects of the code they are working on, help with git/github, our
review process, or the social expectations and norms of the community.  The
goal is that by providing this support to first time contributors we will
retain more of them as regular contributors and then maintainers.

The issue tracker is important to communication in the project because
it serves as the centralized location for making feature requests,
reporting bugs, identifying major projects to work on, and discussing
priorities.  For this reason, it is important to curate the issue
list, adding labels to issues and closing issues that are resolved or
unresolvable. Triaging issues does not require any particular
expertise in the internals of Matplotlib but is extremely valuable to
the project.  To this end we have created a ``Triage Team'' in the
organization who have power to tag, milestone, and close issues.  In
addition to the direct benefit of improving the issue triage and
freeing the core-developers to spend more time reviewing PRs, this
role will bring more people into the developer community and may
provide a path way to becoming regular contributors and maintainers.


We will work with NumFOCUS to develop metrics and evaluate the efficacy of
these efforts at diversifying our contributor base.

\subsubsection{information dissemination}
\begin{enumerate}
\item weekly call
\item docs
\item github
\item discourse
\item mailing lists
\item pydata events
\item twitter
\end{enumerate}

\subsection{current workflow}

Matplotlib is an established community driven in the ``federation''
model as defined by Nadia Eghbal\cite{eghbal_2020}.  We have a core
group regular maintainers who take responsibility for reviewing and
merging proposed changes to the library.  We strive for consensus and
rely on the judgment of our maintainers.

Matplotlib manages proposing and reviewing contributions to the
library and documentation though a variation on the ``git flow''
process\footnote{\url{https://guides.github.com/introduction/flow/}}
on GitHub documented at
\url{https://matplotlib.org/devel/coding_guide.html}.  A contributor,
either one of our core maintainers or a first time contributor, will
open a ``Pull Request'' from their fork of Matplotlib to propose some
change.  Once the PR is opened continuous integration (CI) is
automatically run to preform a variety of regression and style checks.
In parallel the code is reviewed by the maintainers who either request
changes, which starts a cycle of iteration with the contributor, or
approves.  Once consensus is reached the PR is merged and those
changes will be included in the next release.

The threshold for merging a PR depends on the reviewer judgment of the
risk of the changes.  PRs that only change documentation, which can
not introduce regressions or introduce new features, may be merged by
the first reviewer where as all code changes need to be reviewed and
approved by at least two maintainers.  However in either case a
maintainer may wish to leave a positive review but not merge the PR.
If a maintainer objects and requests changes then the PR will not be
merged until those issues have been addressed.  If consensus can not
be reached the default is the status quo (not merging) and the
decision may fall back to the Deputy Project Lead or Project Lead to
break the log jam.

Matplotlib is cautious about making backwards-incompatible change that
intentionally break users existing code.  While in an ideal world,
future versions of the library would always be 100\% backwards
compatible, sometimes we do need to make incompatible changes.  As
part of the review process we check that there if there are any API
changes and if so that they are well justified and documented.  When
technically possible we try to provide user-visible warnings the
version before we actually implement the breaking change.  This gives
us a window of time for users to either adapt to the change or to
communicate to us that they can not adapt so we can reconsider the
change.  Given this high barrier to changing or removing behavior we
are careful to make sure that any new API we add to the library is
well thought out and complete because once we have released a version
of the library with that code it is hard to take it back.  These
considerations together are important enough that we have a Deputy
Project Lead responsible for API consistency.

This process works well for incremental contributions and bug fixes,
new features or bigger changes are typically discussed before
significant work is done.  In many cases if the feature does not need
to be in the core library we encourage contributors to create a new
stand-alone project.  This has several advantages including the giving
the author more control, allows them to iterate faster than our
relatively slow 6 month release schedule, and gives them greater
flexibility to change their API after initial


\newpage
% Here's how I get references.
% needed for AAS citation

\def\ref@jnl#1{{\rm#1}}

\def\aj{\ref@jnl{AJ}}                   % Astronomical Journal
\def\actaa{\ref@jnl{Acta Astron.}}      % Acta Astronomica
\def\araa{\ref@jnl{ARA\&A}}             % Annual Review of Astron and Astrophys
\def\apj{\ref@jnl{ApJ}}                 % Astrophysical Journal
\def\apjl{\ref@jnl{ApJ}}                % Astrophysical Journal, Letters
\def\apjs{\ref@jnl{ApJS}}               % Astrophysical Journal, Supplement
\def\ao{\ref@jnl{Appl.~Opt.}}           % Applied Optics
\def\apss{\ref@jnl{Ap\&SS}}             % Astrophysics and Space Science
\def\aap{\ref@jnl{A\&A}}                % Astronomy and Astrophysics
\def\aapr{\ref@jnl{A\&A~Rev.}}          % Astronomy and Astrophysics Reviews
\def\aaps{\ref@jnl{A\&AS}}              % Astronomy and Astrophysics, Supplement
\def\azh{\ref@jnl{AZh}}                 % Astronomicheskii Zhurnal
\def\baas{\ref@jnl{BAAS}}               % Bulletin of the AAS
\def\bac{\ref@jnl{Bull. astr. Inst. Czechosl.}}
                % Bulletin of the Astronomical Institutes of Czechoslovakia
\def\caa{\ref@jnl{Chinese Astron. Astrophys.}}
                % Chinese Astronomy and Astrophysics
\def\cjaa{\ref@jnl{Chinese J. Astron. Astrophys.}}
                % Chinese Journal of Astronomy and Astrophysics
\def\icarus{\ref@jnl{Icarus}}           % Icarus
\def\jcap{\ref@jnl{J. Cosmology Astropart. Phys.}}
                % Journal of Cosmology and Astroparticle Physics
\def\jrasc{\ref@jnl{JRASC}}             % Journal of the RAS of Canada
\def\memras{\ref@jnl{MmRAS}}            % Memoirs of the RAS
\def\mnras{\ref@jnl{MNRAS}}             % Monthly Notices of the RAS
\def\na{\ref@jnl{New A}}                % New Astronomy
\def\nar{\ref@jnl{New A Rev.}}          % New Astronomy Review
\def\pra{\ref@jnl{Phys.~Rev.~A}}        % Physical Review A: General Physics
\def\prb{\ref@jnl{Phys.~Rev.~B}}        % Physical Review B: Solid State
\def\prc{\ref@jnl{Phys.~Rev.~C}}        % Physical Review C
\def\prd{\ref@jnl{Phys.~Rev.~D}}        % Physical Review D
\def\pre{\ref@jnl{Phys.~Rev.~E}}        % Physical Review E
\def\prl{\ref@jnl{Phys.~Rev.~Lett.}}    % Physical Review Letters
\def\pasa{\ref@jnl{PASA}}               % Publications of the Astron. Soc. of Australia
\def\pasp{\ref@jnl{PASP}}               % Publications of the ASP
\def\pasj{\ref@jnl{PASJ}}               % Publications of the ASJ
\def\rmxaa{\ref@jnl{Rev. Mexicana Astron. Astrofis.}}%
                % Revista Mexicana de Astronomia y Astrofisica
\def\qjras{\ref@jnl{QJRAS}}             % Quarterly Journal of the RAS
\def\skytel{\ref@jnl{S\&T}}             % Sky and Telescope
\def\solphys{\ref@jnl{Sol.~Phys.}}      % Solar Physics
\def\sovast{\ref@jnl{Soviet~Ast.}}      % Soviet Astronomy
\def\ssr{\ref@jnl{Space~Sci.~Rev.}}     % Space Science Reviews
\def\zap{\ref@jnl{ZAp}}                 % Zeitschrift fuer Astrophysik
\def\nat{\ref@jnl{Nature}}              % Nature
\def\iaucirc{\ref@jnl{IAU~Circ.}}       % IAU Cirulars
\def\aplett{\ref@jnl{Astrophys.~Lett.}} % Astrophysics Letters
\def\apspr{\ref@jnl{Astrophys.~Space~Phys.~Res.}}
                % Astrophysics Space Physics Research
\def\bain{\ref@jnl{Bull.~Astron.~Inst.~Netherlands}}
                % Bulletin Astronomical Institute of the Netherlands
\def\fcp{\ref@jnl{Fund.~Cosmic~Phys.}}  % Fundamental Cosmic Physics
\def\gca{\ref@jnl{Geochim.~Cosmochim.~Acta}}   % Geochimica Cosmochimica Acta
\def\grl{\ref@jnl{Geophys.~Res.~Lett.}} % Geophysics Research Letters
\def\jcp{\ref@jnl{J.~Chem.~Phys.}}      % Journal of Chemical Physics
\def\jgr{\ref@jnl{J.~Geophys.~Res.}}    % Journal of Geophysics Research
\def\jqsrt{\ref@jnl{J.~Quant.~Spec.~Radiat.~Transf.}}
                % Journal of Quantitiative Spectroscopy and Radiative Transfer
\def\memsai{\ref@jnl{Mem.~Soc.~Astron.~Italiana}}
                % Mem. Societa Astronomica Italiana
\def\nphysa{\ref@jnl{Nucl.~Phys.~A}}   % Nuclear Physics A
\def\physrep{\ref@jnl{Phys.~Rep.}}   % Physics Reports
\def\physscr{\ref@jnl{Phys.~Scr}}   % Physica Scripta
\def\planss{\ref@jnl{Planet.~Space~Sci.}}   % Planetary Space Science
\def\procspie{\ref@jnl{Proc.~SPIE}}   % Proceedings of the SPIE

\let\astap=\aap
\let\apjlett=\apjl
\let\apjsupp=\apjs
\let\applopt=\ao
\setcounter{page}{1}
\bibliography{mpl_cartopy.bib}

\newpage
\section{Data Management Plan}
\setcounter{page}{1}

Matplotlib is a software library and does not produce any scientific
data as defined in E.1.2 that needs to be preserved.

Matplotlib is currently developed in the open on GitHub and is
released under a permissive license (the Matplotlib license which is a
derivative of the PSF license and compatible with BSD-3).  All work
done on Matplotlib as part of this grant will be done through the
current workflow, will be publicly available, and released under the
same license.  Matplotlib uses git for version control, thus every
developer has the full history on their computer which provides
significant redundancy.  Tagged releases of the software are published
to pypi (in both source and binary forms).  In addition, Anaconada,
macports, homebrew, and all major Linux distributions independently
build, package, and host Matplotlib.  User facing documentation is built
and hosted at \url{https://matplotlib.org}.

Cartopy is currently developed in the open on GitHub and is released
under the LGPL.  All work done on Matplotlib as part of this grant
will be done through the current workflow, will be publicly available,
and released under the same license. Cartopy uses git for version
control, thus every developer has the full history on their computer
which provides significant redundancy.

%do we need something about shape files / map tiles?

Any new libraries created as part of the ROSES award will be developed
in the open on GitHub and will released under a BSD-3 license.

% do we need this?
Any incidental work on other software packages, either upstream or
downstream of Matplotlib and Cartopy, will have to follow the license
and development of process of those projects.


\newpage
\section{Biographical Sketches}
\setcounter{page}{1}
\subsection{Principal Investigator}
\newpage
\subsection{Co-Investigator}

% This line gets the space in TOC right.
\addtocontents{toc}{\protect\vspace{12pt}}
\newpage
\section{Table of Personnel and Work Effort}
\setcounter{page}{1}

\newpage
\section{Current and Pending Support}
\setcounter{page}{1}

\newpage
\section{Budget Justification}
\setcounter{page}{1}

Due to being an established community driven project most regular
contributors have established and distinct primary institutions.  For
this reason significant portions of the budget will need to be
subcontracted to the primary institutions of key individuals who are
uniquely qualified for this work.

\newpage
\section{Facilities and Equipment}
\setcounter{page}{1}

None

\newpage
\section{Detailed Budget}
\setcounter{page}{1}

\begin{itemize}
\item computer + monitor for RSE (2-3k\$)
\item 6-8x conference travel (RSE to scipy and pydata every year + 2
  trips between May / Caswell)
\item 2x trip to JPL
\item 1.5k other IT expenses (discourse, CI, more mac stadium, hosting, ...)
\end{itemize}




\end{document}
